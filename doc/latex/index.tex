Some C/\+C++ compilers are not implementing all of C++11 and above standard, it's often lacking the concurrency features that the standard brings. These compilers will at some point be fixed and we were therefore looking at a way to reduce the effort of switching from a specific implementation to the C++11 standard one.

This projetc is the resulting code.

Of course, this library is a replacement of C++11 features, it is best to use the standard implementation if your compiler support it.

To use this library\+: ``` configure make make install ```

Install moves files into your system's default localtion for headers and libraries (often /usr/local/include and /usr/local/lib). Use this command to change install target directory\+: ``` configure --prefix=/usr/local ```

\href{http://herbertkoelman.github.io/cpp-pthread/doc/html/}{\tt Doxygen documentation} can be generated with this command. I hope this help make things easier to use and understand. ``` make doxygen ```

\begin{quote}
Doxygen can be downloaded \href{http://www.stack.nl/~dimitri/doxygen/index.html}{\tt here}. \end{quote}


The {\ttfamily make} target {\ttfamily pkg} will produce au tar.\+gz that can be distributed.

\subsubsection*{How to use it}

Once compiled and installed in a location that suites you, use your compiler options to reference the headers and the library directory. In almoast all casses you can\+:
\begin{DoxyItemize}
\item include {\ttfamily \#include \char`\"{}pthread/phtread.\+hpp\char`\"{}} in your code to replace of the standard includes.
\item comment out or the c++11 standard includes in your code
\item replace {\ttfamily std} namespace with {\ttfamily pthread} ( std\+::condition\+\_\+variable becomes \hyperlink{classpthread_1_1condition__variable}{pthread\+::condition\+\_\+variable}, etc)
\end{DoxyItemize}

Sample code can be found in the {\ttfamily tests} directory. To use it, run the following commands\+: ``` cd tests ./configure ./make ```

\subsubsection*{Usefull links}

\paragraph*{Memory management on A\+I\+X}

Memory management on A\+I\+X is quite sophisticated making it possible to fine tuned very precisely the way your program uses memory. Consider using these compiler/linker options when using pthreads\+:
\begin{DoxyItemize}
\item -\/bmaxdata\+:0x\+N0000000 this option activates the large memory model, N is a number in the range of \mbox{[}1-\/8\mbox{]}.
\item -\/bmaxmem=-\/1 this option tell the compiler to use as much memory it needs (usefull when -\/\+O option is used).
\end{DoxyItemize}

Thread stack size\+:
\begin{DoxyItemize}
\item 32bits programs allocate 96\+K\+B per thread on the program's heap.
\item 64bits programs allocate 192\+K\+B per thread on the program's heap.
\end{DoxyItemize}

On many Linux implementations and on Mac O\+S X the stack size is defaulted to 8\+M\+B. You may consider setting this as a default.

More detailed information can be found in this \href{http://www.redbooks.ibm.com/redbooks/pdfs/sg245674.pdf}{\tt Red\+Book} (chapter 8).

\paragraph*{project links}


\begin{DoxyItemize}
\item \href{https://github.com/HerbertKoelman/cpp-pthread}{\tt project's home}
\item \href{http://herbertkoelman.github.io/cpp-pthread/doc/html/}{\tt project's doxygen}
\end{DoxyItemize}

\paragraph*{other}


\begin{DoxyItemize}
\item P\+O\+S\+I\+X Threads \href{http://pubs.opengroup.org/onlinepubs/007908799/xsh/threads.html}{\tt documentation}
\item \href{http://en.cppreference.com/w/cpp/thread/thread}{\tt std\+::thread} implementation we try to mimic
\item \href{http://en.cppreference.com/w/cpp/thread/lock_guard/lock_guard}{\tt std\+::lock\+\_\+guard} implementation we try to mimic
\item \href{http://en.cppreference.com/w/cpp/thread/mutex}{\tt std\+::mutex} implementation we try to mimic
\item \href{http://en.cppreference.com/w/cpp/thread/condition_variable}{\tt std\+::condition\+\_\+variable} implementation we try to mimic
\end{DoxyItemize}

\subsubsection*{misc}


\begin{DoxyItemize}
\item author herbert koelman
\item github \href{https://github.com/HerbertKoelman/cpp-pthread}{\tt cpp-\/pthread} 
\end{DoxyItemize}